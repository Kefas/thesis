\chapter{Opis implementacji}
\label{cha:uzywaneTechnologie}

\section{Opis użytych technologii i narzędzi}
\label{sec:technology}
Przy implementacji pracy zostały użyte następujące technologie i narzędzia:
\begin{itemize}
\item Ruby on Rails - jest to framework webowy służący do tworzenia aplikacji internetowych. Został napisany w interpretowanym skryptowym języku Ruby. Ruby on Rails wpisuje się w architekturę Model View Controller (MVC), dostarczając podstawowe struktury dla bazy danych, serwisów oraz stron internetowych. Domyślnym formatem wymiany danych w tym frameworku jest JSON oraz XML. Przy wyświetlaniu stron zaleca się korzystanie z szablonów ERB lub HAML, JavaScript oraz SASS ( jest to język podobny do CSS cechujący się znacznie większą przejrzystością oraz mniejszą ilością kodu). Ruby on Rails w bardzo prosty sposób umożliwa dołączenie do aplikacji dodatkowych, gotowych bibliotek nazywanych gemami. Nadrzędnym celem twórców tego frameworku było wprowadzenie zasad DRY (Don\'t repeat yourself - Nie powtarzaj się) oraz CoC (Convention over Configuration - konwencja ponad konfiguracją). Dzięki kierowania się tymi zasadami kod, który powstaje podczas pisania aplikacji cechuję się wysoką przejrzystością oraz zwięzłością.\cite{rails}
\item Javascript + JQuery + Google Maps API
\item RubyMine IDE
\end{itemize}

\section{Autoryzacja i autentykacja}
\label{sec:autoryzacjaAutentykacja}
