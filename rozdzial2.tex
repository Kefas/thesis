\chapter{Analiza wymagań}
\label{cha:inzyneriaWymagan}
Nadrzędnym celem tworzonych systemów informatycznych jest realizacja wymagań klienta. W przeciwnym razie końcowy użytkownik nie będzie zainteresowany ich odbiorem. Najważniejszą rzeczą we wstępnej fazie procesu tworzenia oprogramowania jest dbałość o zidentyfikowanie właściwych wymagań dla danego problemu.

\section{Zidentyfikowanie użytkowników aplikacji}
\label{sec:uzytkownicy}
Użytkownicy systemu mogą zostać podzielieni na dwie grupy:
\begin{itemize}
\item niezalogowanych
\item zalogowanych
\end{itemize}
System udostępnia pewne funkcjonalności osobom niezarejestrowanym w celu zachęcenia ich do korzystania z serwisu. Zezwolenie na używanie pewnych funkcjonalności serwisu jest sposobem na zachęcenie użytkowników do korzystania z aplikacji. Bardzo często zdarza się, że osoba która nie może przetestować bez rejestracji czy serwis będzie odpowiadał jej potrzebom, nie będzie skłonna podać swoich danych osobowych.

\section{Wymagania funkcjonalne}
\label{sec:wymaganiaFunkcjonalne}
Analiza wymagań funkcjonalnych umożliwia zidentyfikowanie i opisanie pożądanego zachowania systemu. Zgodnie z jedną z definicji, 
\quotes{wymaganie funkcjonalne to „stwierdzenie, jakie usługi ma oferować system, jak ma reagować na określone dane wejściowe oraz jak ma się zachowywać w określonych sytuacjach. W niektórych wypadkach wymagania funkcjonalne określają, czego system nie powinien robić}.\\ Wymaganie funkcjonalne które zostały zidentyfikowane dla użytkowników niezalogowanych:
\begin{itemize}
\item System umożliwia zarejestrowanie się
\item System umożliwia zalogowanie się 
\item System umożliwia przeglądanie ofert
\item System umożliwia używanie filtrów
\end{itemize}
Wymaganie funkcjonane, które zostały zidentyfikowane dla użytkowników zalogowanych:
\begin{itemize}
\item System umożliwia wylogowanie się
\item System umożliwia dodawanie ofert do obserwowanych
\item System umożliwia zapisanie filtrów
\item System umożliwa przeglądanie obserwowanych ofert
\item System umożliwa dodanie oferty 
\end{itemize}
System ponadto powinien:
\begin{itemize}
\item Periodyczne wyszukiwać spersonalizowane oferty dla użytkowników
\item Wysyłać maile do użytkowników z wynikami wyszukiwania
\item Zliczać odsłony oferty
\end{itemize}

\section{Wymagania niefunkcjonalne}
\label{sec;wymaganiaNiefunkcjonalne}
Analiza wymagań niefunkcjonalnych pozwala na określenie kryteriów, które będą brane pod uwagę przy ocenianiu poprawności działania danego systemu. \\
Wymaganie niefunkconalne które zostały określone dla omawianego systemu:
\begin{itemize}
\item System powinien być niezawodny
\item System powinien zwracać wyniki niemal natychmiast
\item System powinien umożliwać dodawanie nowych języków
\item System powinien być bezpieczny
\item System powinien być intuicyjny w użytkowaniu
\end{itemize}
