\chapter{Wprowadzenie}
\label{cha:wprowadzenie}

%---------------------------------------------------------------------------

\section{Cel pracy}
\label{sec:celPracy}
Celem pracy jest zaprojektowanie oraz implementacja aplikacji internetowej umożliwiającej wyszukiwanie oraz dodawanie ogłoszeń dotyczących wynajmu oraz sprzedaży nieruchomości. Tworzony system będzie umożliwiał osobom szukającym lokum w danym mieście przeglądanie ofert oraz korzystanie z szerokiego wachlarza filtrów ułatwiających szukanie idealnego mieszkania lub domu. Aplikacja będzie również oferowała osobom posiadającym nieruchomości, stworzenie ogłoszenia korzystając z prostego i intuicyjnego kreatora ogłoszeń. Ponadto system będzie oferował zarejestrowanym użytkownikom możliwość otrzymywania  spersonalizowanych powiadomień w przypadku pojawienia się nowych ogłoszeń, które spełniają wymagane przez użytkownika kryteria. Dodatkowo aplikacja będzie umożliwiała nawiązanie kontaktu z osobą, która zamieściła ogłoszenie w serwisie. Ze strony technicznej moim celem jest stworzenie prostego w obsłudze i intuicyjnego serwisu, który będzie zrozumiały również dla mniej zaawansowanych użytkowników komputerów, zachowując jednakże określone wymagania dotyczące ogłoszeń, żeby umożliwić użytkownikom dokładne wyszukiwanie ofert.

%---------------------------------------------------------------------------

\section{Zawartość pracy}
\label{sec:zawartoscPracy}
Struktura pracy jest następująca:
\begin{itemize}
\item Pierwszy rozdział stanowi wprowadzanie, definiując przedmiot pracy, jej zawartość,  motyw wyboru tematu oraz analizę podobnych serwisów. 
\item Rozdział drugi przedstawia wymagania biznesowe oraz funkcjonalne i niefunkcjonalne które zdefiniowałem przy procesie projektowania aplikacji. 
\item Rozdział trzeci jest szczegółową analizą przeprowadzoną na podstawie wymagań zdefiniowanych w poprzednim rozdziale. 
\item W rozdziale czwartym są przedstawione konkretne rozwiązania z dziedziny projektowania aplikacji. Opisana jest architektura oraz koncept intefejsu graficznego oraz restowego. W tym rozdziale zawarte są również założenia z dziedziny bezpieczeństwa systemu. 
\item Rozdział piąty stanowi opis używanych technologii. 
\item W szóstym rozdziale jest opisane działanie funkcjonalności dostarczonej aplikacji. 
\item Siódmy rozdział stanowi podsumowanie pracy w którym są zawarte wnioski.
\end{itemize}

%---------------------------------------------------------------------------

\section{Analiza podobnych serwisów}
\label{sec:analizaSerwisow}
